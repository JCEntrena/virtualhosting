%%%%
% Modificaci�n de una plantilla de Latex para adaptarla al castellano.
%%%

%%%%%%%%%%%%%%%%%%%%%%%%%%%%%%%%%%%%%%%%%
% Thin Sectioned Essay
% LaTeX Template
% Version 1.0 (3/8/13)
%
% This template has been downloaded from:
% http://www.LaTeXTemplates.com
%
% Original Author:
% Nicolas Diaz (nsdiaz@uc.cl) with extensive modifications by:
% Vel (vel@latextemplates.com)
%
% License:
% CC BY-NC-SA 3.0 (http://creativecommons.org/licenses/by-nc-sa/3.0/)
%
%%%%%%%%%%%%%%%%%%%%%%%%%%%%%%%%%%%%%%%%%

%----------------------------------------------------------------------------------------
%	PACKAGES AND OTHER DOCUMENT CONFIGURATIONS
%----------------------------------------------------------------------------------------

\documentclass[a4paper, 10pt]{article} % Font size (can be 10pt, 11pt or 12pt) and paper size (remove a4paper for US letter paper)

\usepackage[protrusion=true,expansion=true]{microtype} % Better typography
\usepackage{graphicx} % Required for including pictures
\usepackage[usenames,dvipsnames]{color} % Coloring code
\usepackage{wrapfig} % Allows in-line images
\usepackage[utf8]{inputenc}
\usepackage{enumerate}
\usepackage{enumitem}

\usepackage{geometry}
 \geometry{
 a4paper,
 total={210mm,297mm},
 left=3cm,
 right=3cm,
 top=25mm,
 bottom=25mm,
 }

% Imagenes
\usepackage{graphicx} 

\usepackage{amsmath}
% para importar svg
%\usepackage[generate=all]{svgfig}

% sudo apt-get install texlive-lang-spanish
\usepackage[spanish]{babel} % English language/hyphenation
\selectlanguage{spanish}
% Hay que pelearse con babel-spanish para el alineamiento del punto decimal
\decimalpoint
\usepackage{dcolumn}
\newcolumntype{d}[1]{D{.}{\esperiod}{#1}}
\makeatletter
\addto\shorthandsspanish{\let\esperiod\es@period@code}
\makeatother

\usepackage{longtable}
\usepackage{tabu}
\usepackage{supertabular}

\usepackage{multicol}
\newsavebox\ltmcbox

% Símbolos matemáticos
\usepackage{amssymb}
\let\oldemptyset\emptyset
\let\emptyset\varnothing

% Fuente Arial
\renewcommand{\rmdefault}{phv} % Arial
\renewcommand{\sfdefault}{phv} % Arial



%URL's 
\usepackage{url}

\usepackage[section]{placeins} % Para gr�ficas en su secci�n.
\usepackage{mathpazo} % Use the Palatino font
\usepackage[T1]{fontenc} % Required for accented characters
\newenvironment{allintypewriter}{\ttfamily}{\par}
\setlength{\parindent}{0pt}
\parskip=8pt
\linespread{1.05} % Change line spacing here, Palatino benefits from a slight increase by default

\makeatletter
\renewcommand\@biblabel[1]{\textbf{#1.}} % Change the square brackets for each bibliography item from '[1]' to '1.'
\renewcommand{\@listI}{\itemsep=0pt} % Reduce the space between items in the itemize and enumerate environments and the bibliography
\newcommand{\imagen}[2]{\begin{center} \includegraphics[width=90mm]{#1} \\#2 \end{center}}


\usepackage[hidelinks]{hyperref}

  % Para las enumeraciones anidadas y sus referencias, basado en http://stackoverflow.com/questions/691351/how-to-customize-references-to-sublists-in-latex
  \renewcommand{\theenumi}{\arabic{enumi}.}
  \renewcommand{\theenumii}{\arabic{enumii}}
  \renewcommand{\theenumiii}{\arabic{enumiii}}
  
  \renewcommand{\labelenumi}{\theenumi}
  \renewcommand{\labelenumii}{\theenumi\theenumii.}
  \renewcommand{\labelenumiii}{\theenumi\theenumii.\theenumiii.}
  
  \makeatletter
  \renewcommand{\p@enumii}{\theenumi}
  \renewcommand{\p@enumiii}{\theenumi\theenumii.}
%------------------------------------------------
%	TITLE
%------------------------------------------------

\title{\textbf{Hosting Virtual}} % Title

\date{\today} % Date

%------------------------------------------------

\begin{document}

\maketitle
\tableofcontents
\pagebreak

\section{Resumen}

En este trabajo vamos a tratar sobre el hosting virtual y su implementación en un servidor apache. Inicialmente presentaremos el concepto de hosting virtual y sus diferentes opciones y usos, continuando con el servidor apache, del cual trataremos los conceptos básicos y enlazaremos con el hosting virtual. Analizaremos y comprobaremos las diferentes opciones que nos ofrece apache para ofrecer este tipo de servicio, explicando las alternativas y las ventajas que nos ofrece cada una de ellas. Trataremos también distintos aspectos a tener en cuenta a la hora de usar hosting virtual en un servidor, tales como la seguridad, y veremos qué opciones nos permite apache para configurar dichos aspectos. 

\section{Memoria}

Memoria. 

\subsection{Introducción}

Una introducción al tema. Máximo, 2 páginas. Hablaremos de Virtual Hosting y Apache. 

\subsection{Virtual Hosting}

El concepto de Virtual Hosting se refiere al método para alojar múltiples dominios en un único servidor, lo que permite a dicho servidor compartir su funcionalidad entre los distintos dominios, como por ejemplo la memoria o el procesador. Uno de sus principales usos es el almacenamiento web compartido, un servicio donde cada sitio web ocupa una parte del servidor, que está contectado a internet. Es la forma de almacenamiento más económica, pues permite a todo aquel que quiera tener una página web alojarla a un precio mucho más barato ya que se reparten los gastos de mantenimiento y alquiler del servidor entre todos los usuarios. 

Existen distintos tipos de hosting virtual, de los cuales destacan dos: basado en nombre (name-based) y en IP (IP-based), aunque también existe una versión basada en puertos(port-based), si bien es más extraña. Vamos a hablar un poco de ellas en detalle: 

Name-based: Esta opción almacena hosts con distintos nombre pero que comparten la misma IP. De esta forma, podríamos tener dos páginas, www.ejemplo.com y www.ejemplo.org compartiendo IP pero distintas entre sí (y con almacenamientos distintos). En esta opción se nos presenta un problema con las DNS, ya que si estas no funcionan debidamente resulta imposible acceder a la página web, aun conociendo la IP. Por norma general, si intentamos acceder directamente mediante la IP el servidor responderá con una página web por defecto, que usualmente no es a la que el usuario quiere accerder. \\
Existen ciertos problemas de seguridad con esta opción, pero los debatiremos más adelante. 

IP-based: En esta opción cada página web tendrá una IP única para sí misma. Es menos problemático en cuanto al acceso, pues cada web es distinta a las demás y se gestiona por separado, aunque aumenta un poco la carga del servidor, que tiene que escuchar distintas IP's, si bien esto no es significativo en gran medida. 

Estas dos técnicas pueden ser combinadas dentro de un servidor, no teniendo que restringirse únicamente a una de las dos. 

\subsection{Apache}

Hablar de Apache.

Apache es un servidor web HTTP de código abierto, siendo actualmente el más utilizado en todo el mundo.\footnote{Podemos ver estadísticas de uso en \url{http://news.netcraft.com/archives/2015/04/20/april-2015-web-server-survey.html\#more-18965}} Actualmente se encuentra en su versión 2.4.12.

Configuración de apache. He cambiado la dirección de correo donde quiero que se me envíen las notificaciones. 


\subsection{Trabajo}

¿Cómo configurar Apache para poder hacer hosting virtual? 

Mirar la documentación de apache, http://httpd.apache.org/docs/2.4/en/vhosts/name-based.html

Podemos hablar de la configuración de apache en general y luego centrarnos en el hosting virtual. Hablar de qué hay que tocar en los archivos de configuración puede ser buena idea, y quizás hacer alguna prueba en local. (Puede valer como inicio de la presentación?) 

Diferenciación entre basados en nombre y basados en IP. 
	- Definición de cada uno de ellos
	- Cómo los trata apache. 
	- Ejemplos de cada uno, abordando todas las posibilidades. 

Hablar de la seguridad en apache y como afecta al virtual hosting en particular. Proponer soluciones para problemas que se puedan dar. 

Permitir multiples sitios seguros: Server Name Indication 


\end{document}