%%%%
% Modificaci�n de una plantilla de Latex para adaptarla al castellano.
%%%

%%%%%%%%%%%%%%%%%%%%%%%%%%%%%%%%%%%%%%%%%
% Thin Sectioned Essay
% LaTeX Template
% Version 1.0 (3/8/13)
%
% This template has been downloaded from:
% http://www.LaTeXTemplates.com
%
% Original Author:
% Nicolas Diaz (nsdiaz@uc.cl) with extensive modifications by:
% Vel (vel@latextemplates.com)
%
% License:
% CC BY-NC-SA 3.0 (http://creativecommons.org/licenses/by-nc-sa/3.0/)
%
%%%%%%%%%%%%%%%%%%%%%%%%%%%%%%%%%%%%%%%%%

%----------------------------------------------------------------------------------------
%	PACKAGES AND OTHER DOCUMENT CONFIGURATIONS
%----------------------------------------------------------------------------------------

\documentclass[a4paper, 10pt]{article} % Font size (can be 10pt, 11pt or 12pt) and paper size (remove a4paper for US letter paper)

\usepackage[protrusion=true,expansion=true]{microtype} % Better typography
\usepackage{graphicx} % Required for including pictures
\usepackage[usenames,dvipsnames]{color} % Coloring code
\usepackage{wrapfig} % Allows in-line images
\usepackage[utf8]{inputenc}
\usepackage{enumerate}
\usepackage{enumitem}

\usepackage{geometry}
 \geometry{
 a4paper,
 total={210mm,297mm},
 left=3cm,
 right=3cm,
 top=25mm,
 bottom=25mm,
 }

% Imagenes
\usepackage{graphicx} 

\usepackage{amsmath}
% para importar svg
%\usepackage[generate=all]{svgfig}

% sudo apt-get install texlive-lang-spanish
\usepackage[spanish]{babel} % English language/hyphenation
\selectlanguage{spanish}
% Hay que pelearse con babel-spanish para el alineamiento del punto decimal
\decimalpoint
\usepackage{dcolumn}
\newcolumntype{d}[1]{D{.}{\esperiod}{#1}}
\makeatletter
\addto\shorthandsspanish{\let\esperiod\es@period@code}
\makeatother

\usepackage{longtable}
\usepackage{tabu}
\usepackage{supertabular}

\usepackage{multicol}
\newsavebox\ltmcbox

% Símbolos matemáticos
\usepackage{amssymb}
\let\oldemptyset\emptyset
\let\emptyset\varnothing

% Fuente Arial
\renewcommand{\rmdefault}{phv} % Arial
\renewcommand{\sfdefault}{phv} % Arial



%URL's 
\usepackage{url}

\usepackage[section]{placeins} % Para gr�ficas en su secci�n.
\usepackage{mathpazo} % Use the Palatino font
\usepackage[T1]{fontenc} % Required for accented characters
\newenvironment{allintypewriter}{\ttfamily}{\par}
\setlength{\parindent}{0pt}
\parskip=8pt
\linespread{1.05} % Change line spacing here, Palatino benefits from a slight increase by default

\makeatletter
\renewcommand\@biblabel[1]{\textbf{#1.}} % Change the square brackets for each bibliography item from '[1]' to '1.'
\renewcommand{\@listI}{\itemsep=0pt} % Reduce the space between items in the itemize and enumerate environments and the bibliography
\newcommand{\imagen}[2]{\begin{center} \includegraphics[width=90mm]{#1} \\#2 \end{center}}


\usepackage[hidelinks]{hyperref}

  % Para las enumeraciones anidadas y sus referencias, basado en http://stackoverflow.com/questions/691351/how-to-customize-references-to-sublists-in-latex
  \renewcommand{\theenumi}{\arabic{enumi}.}
  \renewcommand{\theenumii}{\arabic{enumii}}
  \renewcommand{\theenumiii}{\arabic{enumiii}}
  
  \renewcommand{\labelenumi}{\theenumi}
  \renewcommand{\labelenumii}{\theenumi\theenumii.}
  \renewcommand{\labelenumiii}{\theenumi\theenumii.\theenumiii.}
  
  \makeatletter
  \renewcommand{\p@enumii}{\theenumi}
  \renewcommand{\p@enumiii}{\theenumi\theenumii.}
%------------------------------------------------
%	TITLE
%------------------------------------------------

\title{\textbf{Hosting Virtual}} % Title

\date{\today} % Date

%------------------------------------------------

\begin{document}

\maketitle
\tableofcontents
\pagebreak

\section{Resumen}

Un resumen de entre 5 y 15 líneas del trabajo. 


Links en Windows. 
http://en.wikipedia.org/wiki/Shared_web_hosting_service
http://es.wikipedia.org/wiki/Server_Name_Indication
http://blog.arkabytes.com/linux/servidor-casero-servidor-web-apache/
http://httpd.apache.org/docs/2.4/en/vhosts/name-based.html
http://httpd.apache.org/docs/2.4/vhosts/examples.html

\section{Memoria}

Memoria. 

\section{Introducción}

Una introducción al tema. Máximo, 2 páginas. Hablaremos de Virtual Hosting y Apache. 

\subsection{Virtual Hosting}

El concepto de Virtual Hosting se refiere al método para alojar múltiples dominios en un único servidor, lo que permite a dicho servidor compartir su funcionalidad entre los distintos dominios, como por ejemplo la memoria o el procesador. 

\subsection{Apache}

Hablar de Apache. 

\section{Trabajo}

¿Cómo configurar Apache para poder hacer hosting virtual? 

Mirar la documentación de apache, http://httpd.apache.org/docs/2.4/en/vhosts/name-based.html

Podemos hablar de la configuración de apache en general y luego centrarnos en el hosting virtual. Hablar de qué hay que tocar en los archivos de configuración puede ser buena idea, y quizás hacer alguna prueba en local. (Puede valer como inicio de la presentación?) 

Diferenciación entre basados en nombre y basados en IP. 
	- Definición de cada uno de ellos
	- Cómo los trata apache. 
	- Ejemplos de cada uno, abordando todas las posibilidades. 

Hablar de la seguridad en apache y como afecta al virtual hosting en particular. Proponer soluciones para problemas que se puedan dar. 

Permitir multiples sitios seguros: Server Name Indication 


\end{document}